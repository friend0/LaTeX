\documentclass[Nomencl]{SelimArticle}
%!TEX root = main.tex

%%%%%%%%%%%%%%%%%%%%%%%%%%%%%%%
% Additional Packages/Options %
%%%%%%%%%%%%%%%%%%%%%%%%%%%%%%%
% \setlist{nosep}
\hypersetup{hidelinks}

%%%%%%%%%%%%%%%%%
% Title Options %
%%%%%%%%%%%%%%%%%
\usepackage{mypage}
\school{UC Santa Cruz}
\course{CE241}
\coursenum{Course Code}
%Add \\[0.3cm] for new line.
\title{Closed Loop Position Control for Characterizing Noise in Motion Capture Systems}
\student{Ryan \textsc{Rodriguez}}
\studentnum{1336732}
\date{\today}

%%%%%%%%%%%%%%%%%%%%%%%%%
% Additional Formatting %
%%%%%%%%%%%%%%%%%%%%%%%%%
%Horizontal line below section.
\sectionfont{\sectionrule{0pt}{0pt}{-5pt}{0.8pt}}
%Section numbering depth. Value of 2 means numbering ends with subsections.
\setcounter{secnumdepth}{3}
%Table of contents section depth. Same as above.
\setcounter{tocdepth}{2}
\numberwithin{equation}{section}
\numberwithin{figure}{section}
\newcommand{\ra}[1]{\renewcommand{\arraystretch}{#1}}

\begin{document}
\mytitlepage
\tableofcontents
\newpage
\printnomenclature % Use makeidx.bat followed by two pdflatex reruns.
\newpage
% Begin writing here.

\section{Motivation}
Over the past several years, the application of infrared motion captures systems to the design and validation of control laws for a wide variety of problems. Although the server applications that accompany these systems include faculties for determining mean error based on a simple calibration step, it may often be the case that control engineers may want to verify their 'ground truth' based on more empirical evidence.

The focus of my project was on the design of  a closed loop positioning platform to be used in characterization of motion capture system noise. This was done using aluminum extrusion coupled to stepper belt-drives, with the addition of closed loop control in the form of a 'flying-fader' a type of linear motorized potentiometer commonly used in audio control panel applications. 

The components are pictured in figure REF and REF.

\section{Analysis}
Naturally, the design of this controller began with some mathematical analysis to specify the physical system to be controlled. In the case of the flying fader, the physical mechanism consists of a DC motor coupled to a belt with a pulley. The belt, stretched taught by pulley's on either end, one passive, one active, serves as an energy transfer mechanism to a small tab attached to the belt for positioning some object, typically a sliding knob. 

The dynamics for the motor are given as follows:

The dynamics of the belt/pulley system are given as follows:

The system transfer function is given as follows:


\section{Gallery}

\bibliographystyle{IEEEtran}
\bibliography{dabib}
\end{document}